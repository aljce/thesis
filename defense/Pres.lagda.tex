\documentclass{beamer}
\usepackage{amsmath}
\usepackage{amssymb}
\usepackage{amsthm}
\usepackage{mathbbol}
\usepackage{mathtools}
\usepackage{stmaryrd}
\DeclarePairedDelimiter\Parens{\lparen}{\rparen}
\DeclarePairedDelimiter\Floor{\lfloor}{\rfloor}
\usepackage{gauss}
\usepackage{relsize}

\usepackage{xcolor}

\usepackage{algorithm}
\usepackage[noend]{algpseudocode}

\usepackage{url}
\usepackage{hyperref}
\usepackage[T1]{fontenc}
\usepackage[utf8]{inputenc}
\usepackage[links]{agda}
\usepackage{newunicodechar}
\newunicodechar{λ}{\ensuremath{\mathnormal\lambda}}
\newunicodechar{→}{\ensuremath{\mathnormal\to}}
\newunicodechar{∀}{\ensuremath{\mathnormal\forall}}
\newunicodechar{𝔹}{\ensuremath{\mathbb{B}}}
\newunicodechar{ℕ}{\ensuremath{\mathbb{N}}}
\newunicodechar{₁}{\ensuremath{{}_1}}
\newunicodechar{¹}{\ensuremath{{}^1}}
\newunicodechar{₂}{\ensuremath{{}_2}}
\newunicodechar{²}{\ensuremath{{}^2}}
\newunicodechar{₃}{\ensuremath{{}_3}}
\newunicodechar{³}{\ensuremath{{}^3}}
\newunicodechar{₄}{\ensuremath{{}_4}}
\newunicodechar{⁴}{\ensuremath{{}^4}}
\newunicodechar{ᵇ}{\ensuremath{{}^b}}
\newunicodechar{⁺}{\ensuremath{{}^+}}
\newunicodechar{≡}{\small {\ensuremath{\equiv}}}
\newunicodechar{≢}{\small {\ensuremath{\not\equiv}}}
\newunicodechar{≟}{\small {\ensuremath{\stackrel{?}{=}}}}
\newunicodechar{∙}{\ensuremath{\cdot}}
\newunicodechar{×}{\ensuremath{\times}}
\newunicodechar{ε}{\ensuremath{\varepsilon}}
\newunicodechar{∎}{\ensuremath{\rule{1ex}{1.2ex}}}
\newunicodechar{⟨}{\ensuremath{\langle}}
\newunicodechar{⟩}{\ensuremath{\rangle}}
\newunicodechar{ˡ}{\ensuremath{{}^l}}
\newunicodechar{ʳ}{\ensuremath{{}^r}}
\newunicodechar{∷}{\ensuremath{::}}
\newunicodechar{ⁱ}{\ensuremath{{}^i}}
\newunicodechar{ʰ}{\ensuremath{{}^h}}
\newunicodechar{ℓ}{\ensuremath{\ell}}
\newunicodechar{⁻}{\ensuremath{{}^{-}}}
\newunicodechar{≈}{\ensuremath{\approx}}
\newunicodechar{𝕓}{\ensuremath{\mathbb{b}}}
\newunicodechar{⇓}{\ensuremath{\Downarrow}}
\newunicodechar{⇑}{\ensuremath{\Uparrow}}
\newunicodechar{⇒}{\ensuremath{\Rightarrow}}
\newunicodechar{⇐}{\ensuremath{\Leftarrow}}
\newunicodechar{⟦}{\ensuremath{\llbracket}}
\newunicodechar{⟧}{\ensuremath{\rrbracket}}
\newunicodechar{✓}{\ensuremath{\checkmark}}
\newunicodechar{∣}{\ensuremath{\mid}}
\newunicodechar{∤}{\ensuremath{\nmid}}
\newunicodechar{¬}{\ensuremath{\neg}}
\newunicodechar{⊥}{\ensuremath{\bot}}
\newcommand{\Dotdiv}{%
  \mathbin{%
    \vphantom{+}%
    \text{%
      \mathsurround=0pt%
      \ooalign{%
        \noalign{\kern-.35ex}%
        \hidewidth$\smash{\cdot}$\hidewidth\cr%
        \noalign{\kern.35ex}%
        $-$\cr%
      }%
    }%
  }%
}
\newunicodechar{∸}{\ensuremath{\Dotdiv}}
\newunicodechar{≤}{\ensuremath{\leq}}
\newunicodechar{≱}{\ensuremath{\not\geq}}
\newunicodechar{≮}{\ensuremath{\not<}}
\newunicodechar{≺}{\ensuremath{\prec}}
\newunicodechar{⊂}{\ensuremath{\subset}}
\newunicodechar{∨}{\ensuremath{\wedge}}

\usepackage{pifont}
\newcommand{\cmark}{\text{\ding{51}}}
\newcommand{\xmark}{\text{\ding{55}}}

\newcommand{\Agda}[0]{Agda}

\newcommand{\BigO}[1]{\ensuremath{O\Parens{#1}}}

\newcommand{\Lg}[1]{\ensuremath{\text{lg}\Parens{#1}}}

\setlength{\parindent}{0pt}

\usetheme{metropolis}
\title{
  A Replication of the AKS\\
  Primality Decision Algorithm
}
\date{8 May 2020}
\author{Alice McKean}
\institute{
  The Division of Mathematical and Natural Sciences \\
  Reed College
}

\begin{document}

\maketitle

\begin{frame}{\Agda{} Introduction}
  % Imagine not having to grade any math homework ever again or mathematics journals not
  % needing peer review. This is the world \Agda{} and other \textit{proof assistants}
  % hope to bring about. \\
  \Agda{} is a \textit{proof assistant}. It can verify the correctness of
  arbitrarily complex proofs in higher order logic \cite{agda}. \\
  Its not magic though, it can only perform the most basic proof search. This
  sets it apart from other tools in this area, like SAT solvers. \\
  In this thesis, we will attempt to verify the AKS primality decision procedure
  in \Agda{}.
\end{frame}

\section{Primality Decision Algorithms}

\begin{frame}{Problem Definition}
  In this thesis we investigate \textit{decision algorithms} for the following
  language.
  \begin{align*}
    \text{PRIMES} = \{ \, p \, | \text{ such that } p \text{ is a prime number } \}
  \end{align*}
  An obvious brute force solution follows from the definition of prime numbers.
  Just try all the divisors less than $p$ and if nothing divides $p$ return YES. \\

  This algorithm is linear in the size of the input $\BigO{n}$ we desire an
  algorithm with polynomial run-time in the size of the description of
  the input $\BigO{\Lg{n}^c}$.
\end{frame}
\begin{frame}{Algorithm Characterization}
  These three dimensions are how we will compare primality decision algorithms.
  \begin{itemize}
    \item \textit{Computational Complexity} standard run-time analysis
    \item \textit{Determinism} if the algorithm requires probabilistic primitives
    \item \textit{Conditionality} an algorithm is conditional it's correctness
      depends on unproven conjecture
  \end{itemize}
\end{frame}

\begin{frame}{History}
  Now we will discuss a rough history of the development of primality decision
  algorithms.
  \begin{itemize}
    \item<1-> Miller's Primality Test, deterministic, polynomial time but requires
      \textit{Extended Riemann Hypothesis} \cite{miller}
    \item<2-> Rabin's Primality Test, polynomial time, unconditional but
      non-deterministic \cite{rabin}
    \item<3-> Adleman, Pomerance, and Rumely, deterministic, unconditional and
      sub-exponential
      $\BigO{\Lg{n}^{c \, \text{lg} \, \text{lg} \, \text{lg} \, n}}$ \cite{apr}. 
  \end{itemize}
\end{frame}

\begin{frame}{AKS}
  Agrawal, Kayal,and Saxena created a unconditional, deterministic, polynomial
  time $\BigO{\Lg{n}^{15/2}}$ primality decision algorithm. \cite{aks}. \\

  The correctness proof of this algorithm is comparably simpler to all the
  previous algorithms listed above, requiring only undergraduate levels of number
  theory. This property makes AKS a great candidate for a ``replication'' in
  \Agda{}. \\

  Next we explore some psuedo-code describing AKS adapted from Micheal Smid's
  AKS analysis \cite{smid}.
\end{frame}

\begin{frame}{AKS (Pseudo Code)}
  \scriptsize
  \begin{algorithmic}
    \Function{IsPrime}{$n$}
    \State $r \leftarrow 2$
    \While{true}
      \If{$gcd(r, n) = 1$}
        \State \textbf{return} NO
      \EndIf
      \State $q \leftarrow$ largest prime factor of $r - 1$
      \If{$q \geq 2 \sqrt{r} \, \Lg{n} + 2$ and
        $n^{(r - 1) / q} \not\equiv 1 \text{ mod } r$}
        \State \textbf{break}
      \EndIf
      \State $r \leftarrow$ next prime after $r$
    \EndWhile
    \For{$a \leftarrow 1$ \textbf{to} $\Floor[\big]{2 \sqrt{r} \, \Lg{n}} + 1$}
      \If{$(x - a)^n \not\equiv (x^n - a) \text{ mod } (x^r - 1) \text{ in } \mathbb{Z}_n[x]$}
        \State \textbf{return} NO
      \EndIf
    \EndFor
    \If{$n$ is a perfect power}
      \State \textbf{return} NO
    \Else
      \State \textbf{return} YES
    \EndIf
    \EndFunction
  \end{algorithmic}
\end{frame}

\begin{frame}{Primitive Concepts}
  \begin{center}
    \begin{tabular}{c|c|c}
      Concept & Formalized & In Thesis \\
      \hline
      Define Primality & \cmark & \cmark \\
      Looping Constructs & \cmark & \cmark \\
      Prime Stream & \cmark & \cmark \\
      Square Root & \xmark & \xmark \\
      Binary Logarithm & \cmark & \cmark \\
      Modular Field & \cmark & \xmark \\
      Polynomials & \cmark & \xmark \\
      Polynomial Division & \cmark & \xmark \\
      Exponentiation & \cmark & \cmark \\
      Testing Perfect Powers & \xmark & \xmark \\
    \end{tabular}
  \end{center}
\end{frame}

\section{Formalized Proofs}

\begin{frame}{Type Theory}
The primitive concept underlying Type Theory is the \textit{typing relation},
namely a \textit{term} $a$ has \textit{type} $A$ written $a : A$. \\
In the language of Set Theory this would represent ``a is an element of the set A''
denoted $a \in A$. \\
Types differ from sets in many ways. For instance the perfectly valid set
$\{ \, \text{true}, \, 0, \, \emptyset \, \}$
is inexpressible in Type Theory as types are ``homogeneous''. \\
``Is $b$ a member of $B$ for some pre-existing objects $b$ and $B$?'' is
nonsense in Type Theory. One can not talk about objects without their type.
\end{frame}

\begin{frame}{Booleans}
  \begin{code}[hide]
    open import Relation.Binary.PropositionalEquality
      using (_≡_; _≢_; module ≡-Reasoning)
      renaming (refl to ≡-refl; sym to ≡-sym; cong to ≡-cong)
    open ≡-Reasoning
  \end{code}
  \begin{code}
    data 𝔹 : Set where
      false : 𝔹
      true  : 𝔹

    not : 𝔹 → 𝔹
    not false = true
    not true  = false

    not-example : 𝔹
    not-example = not false
    
    not-involution : ∀ (b : 𝔹) → not (not b) ≡ b
    not-involution false = ≡-refl
    not-involution true = ≡-refl
  \end{code}
\end{frame}

\begin{frame}{Natural Numbers}
  \begin{code}
    data ℕ : Set where
      zero : ℕ
      suc  : ℕ → ℕ

    three : ℕ
    three = suc (suc (suc zero))

    _+_ : ℕ → ℕ → ℕ
    zero + n = n
    suc n + m = suc (n + m)
  \end{code}
\end{frame}

\begin{frame}{Addition is Associative}
  \begin{code}
    Associative : ∀ {A : Set} → (A → A → A) → Set
    Associative {A} _∙_
      = ∀ (x y z : A) → (x ∙ (y ∙ z)) ≡ ((x ∙ y) ∙ z)

    +-assoc : Associative _+_
    +-assoc zero y z = ≡-refl
    +-assoc (suc x) y z = begin
      suc x + (y + z)   ≡⟨⟩
      suc (x + (y + z)) ≡⟨ ≡-cong suc (+-assoc x y z) ⟩
      suc ((x + y) + z) ≡⟨⟩
      (suc (x + y) + z) ≡⟨⟩
      (suc x + y) + z   ∎
  \end{code}
\end{frame}

\begin{frame}[allowframebreaks]{Citations}
  \nocite{gnu-parallel}
  \bibliographystyle{apalike}
  \bibliography{citations}
\end{frame}

\end{document}