\documentclass[./Thesis.tex]{subfiles}
\begin{document}

\chapter{Primality}
\label{chap:primality}

\epigraph{
  3 is prime, 5 is prime, 7 is prime. By induction, all the odd integers are
  prime.
}{\cite{leroy}}
In this chapter we will explore various proof relevant definitions of primality.
Then we will construct a decision procedure for these definitions via the
standard brute force primality algorithm.
\begin{code}[hide]
  module Primality where
  open import AKS.Nat using (ℕ; zero; suc; _+_; _∸_; _*_; _≤_; _<_; lte; _<?_; _≟_)
  open import AKS.Nat using (auto-≤; <⇒≱; n≮n; ≤-<-trans; m≤m+n; ≢⇒¬≟; suc-injective-≤; ≤-refl) renaming (n≤m⇒n<m⊎n≡m to n≤m⇒n<m∨n≡m)
  open import AKS.Nat using (<-irrefl; <-cmp; <-trans; n<1+n; 0<1+n)
  open import AKS.Nat using (Acc; acc; <-well-founded; Interval; [_,_∣_]; ⊂-well-founded; _⊂_; upward)
  open import Data.Nat.Properties using (*-identityˡ; *-comm; *-assoc; *-zeroʳ; *-distribʳ-∸; m+n∸n≡m)
  open import AKS.Nat.Divisibility using (_/_; Euclidean✓; _div_)
  open import Relation.Nullary using (¬_)
  open import Relation.Nullary.Decidable using (False; True; from-yes; from-no)
  open import Relation.Nullary.Negation using (contradiction)
  open import Relation.Binary using (Reflexive; Transitive; Tri)
  open Tri
  open import Relation.Binary.PropositionalEquality
    using (_≡_; _≢_; module ≡-Reasoning)
    renaming (refl to ≡-refl; sym to ≡-sym; cong to ≡-cong; cong₂ to ≡-cong₂)
  open import Data.Empty using (⊥; ⊥-elim)
  open import Data.Unit using (tt)
  open import Data.Sum using (inj₁; inj₂)
  open import Function using (_$_) -- $
  open ≡-Reasoning
\end{code}
\section{Divisibility}
\label{sec:divisibility}
Before we can define primality we require a definition of divisibility. We
follow a standard definition of divisibility, namely a natural number $d$
divides into another natural $a$ iff there exists some $q$ such that $a = q d$.
\begin{code}[hide]
  infix 3 _∣_
\end{code}
\begin{code}
  record _∣_ (d : ℕ) (a : ℕ) : Set where
    constructor divides
    field
      quotient : ℕ
      equality : a ≡ quotient * d

  _∤_ : ℕ → ℕ → Set
  d ∤ a = ¬ (d ∣ a)
\end{code}
In order to make a decision procedure for primality we first require a decision
procedure for divisibility. This would have to be a proof relevant algorithm. An
algorithm that would take two natural numbers and produce either a proof of
divisibility or a proof of non-divisibility. This is no small task, one which
will become easier with the correct lemmas. We first prove that the divisibility
relation is reflexive, that is that for any $n$ the expression
$n \, \AgdaFunction{∣} \, n$ is true. After which we will prove that the
divisibility relation is transitive. 
\begin{code}
  ∣-refl : Reflexive _∣_
  ∣-refl {n} = divides 1 $ begin n ≡⟨ ≡-sym (*-identityˡ n) ⟩ 1 * n ∎
\end{code} % $
\begin{code}
  ∣-trans : Transitive _∣_
  ∣-trans {x} {y} {z} (divides q₁ y≡q₁*x) (divides q₂ z≡q₂*y) = divides (q₂ * q₁) $
    z ≡⟨ z≡q₂*y ⟩
    q₂ * y ≡⟨ ≡-cong (λ t → q₂ * t) y≡q₁*x ⟩
    q₂ * (q₁ * x) ≡⟨ ≡-sym (*-assoc q₂ q₁ x) ⟩
    (q₂ * q₁) * x ∎
\end{code} % $
As \Agda{} is a proof relevant logic, every proof of divisibility materializes
similarly. The user simply specifies what number to multiply the divisor by to
get the dividend. Then they supply a proof that the two numbers do indeed multiply
to the dividend. It the following proof we assume a divisibility relation is
true. We ``consume'' the divisibility relation.
Specifically we consider when $0$ divides another number. The only number $0$
divides is $0$ itself.
\begin{code}
  0∣n⇒n≡0 : ∀ {n} → 0 ∣ n → n ≡ 0
  0∣n⇒n≡0 {n} (divides q n≡q*0) = begin
    n ≡⟨ n≡q*0 ⟩
    q * 0 ≡⟨ *-zeroʳ q ⟩
    0 ∎
\end{code}
The following two lemmas have simple proofs, so we leave the proofs to the
reader. The first lemma shows that the divisibility relation respects equality
in its second argument and the second lemma shows that some number $m$ will
divide a number if it has $m$ as a factor.
\begin{code}
  ∣-respʳ : ∀ {i n m} → n ≡ m → i ∣ n → i ∣ m

  m∣m*n : ∀ {n m} → m ∣ m * n
\end{code}
\begin{code}[hide]
  ∣-respʳ {i} {n} {m} n≡m (divides q n≡q*i) =
    divides q $ begin
      m ≡⟨ ≡-sym n≡m ⟩
      n ≡⟨ n≡q*i ⟩
      q * i ∎

  m∣m*n {n} {m} = divides n $ begin m * n ≡⟨ *-comm m n ⟩ n * m ∎
\end{code}
This proof shows is the ``reverse'' of the idea that if a number $i$ divides two
numbers $m$ and $n$ then it must divide their sum. This because $i$ is a common
factor of the two numbers. If $n + m = q_1 i$ and $m = q_2 i$ then it follows
that $n = (q_1 - q_2) * i$. Note that $q_1 \geq q_2$ as $q_1$ contains $q_2$. This
is important in the following \Agda{} proof as the type $\AgdaDatatype{ℕ}$ does
not contain negative numbers. There is a subtraction like function commonly
called ``monus'' that returns $0$ when the difference is negative. This is
denoted as $\AgdaFunction{∸}$ in \Agda{} and it behaves similarly to subtraction in many
proofs.
\begin{code}
  ∣n+m∣m⇒∣n : ∀ {i n m} → i ∣ n + m → i ∣ m → i ∣ n
  ∣n+m∣m⇒∣n {i} {n} {m} (divides q₁ n+m≡q₁*i) (divides q₂ m≡q₂*i) =
    divides (q₁ ∸ q₂) $ begin
      n               ≡⟨ ≡-sym (m+n∸n≡m n m) ⟩
      (n + m) ∸ m     ≡⟨ ≡-cong₂ (λ x y → x ∸ y) n+m≡q₁*i m≡q₂*i ⟩
      q₁ * i ∸ q₂ * i ≡⟨ ≡-sym (*-distribʳ-∸ i q₁ q₂) ⟩
      (q₁ ∸ q₂) * i   ∎
\end{code} % $
This final lemma reduces many proofs about divisibility into proofs about
inequalities. This lemma occurs multiple times in the following sections and
chapters. The key idea is that if $n$ divides $m$ then it must be less than or
equal to $m$ in order to be a factor. Technically this is false when $m = 0$ as
every number divides $0$. So we just assume $m \neq 0$. This proof illustrates
the importance of choosing definitions correctly. Using the definition
$\exists k. \, n + k = m$ of $n \leq m$ makes this proof almost trivial. An
inductive definition of $n \leq m$ would complicate this proof by a large measure.
\begin{code}
  ∣⇒≤ : ∀ {n m} → m ≢ 0 → n ∣ m → n ≤ m
  ∣⇒≤ {n} {zero} m≢0 (divides q m≡q*n) = contradiction ≡-refl m≢0
  ∣⇒≤ {n} {suc m} m≢0 (divides (suc q) m≡[1+q]*n) =
    lte (q * n) $ begin
      n + q * n ≡⟨⟩
      suc q * n ≡⟨ ≡-sym m≡[1+q]*n ⟩
      suc m     ∎
\end{code} % $
With the lemmas out of the way we have one last bookkeeping task. What is the
correct type of decision procedures? A proof relevant decision procedure should
return not just the truth value of the predicate the user is testing. It should
also return a proof that the predicate is true or a proof that the predicate is
false. These cases are mutually exclusive so we can represent this idea with the
following datatype. Note this type can be used generically in decision producers
over any proposition $P$.
\begin{code}[hide]
  module Decidable where
\end{code}
\begin{code}
    data Dec (P : Set) : Set where
      yes :   P → Dec P
      no  : ¬ P → Dec P
\end{code}
\begin{code}[hide]
  open import Relation.Nullary using (Dec; yes; no)
\end{code}
The full decision procedure begins by checking if the divisor is $0$. If the
divisor is $0$, then we ensure the dividend is $0$ as $0$ only divides $0$.
\begin{code}
  _∣?_ : ∀ (d a : ℕ) → Dec (d ∣ a)
  d ∣? a with d ≟ 0
  d ∣? a | yes ≡-refl with a ≟ 0
  d ∣? a | yes ≡-refl | yes ≡-refl = yes ∣-refl
  d ∣? a | yes ≡-refl | no a≢0 = no λ 0∣a → contradiction (0∣n⇒n≡0 0∣a) a≢0
\end{code}
If the divisor is not $0$, then we can divide the dividend by the divisor. If
the integer remainder of that division is $0$, then the divisor divides the
dividend. Otherwise the divisor can not divide the dividend as the remainder is
strictly less than the divisor.
\begin{code}
  d ∣? a | no d≢0 with (a div d) {≢⇒¬≟ d≢0}
  d ∣? a | no d≢0 | Euclidean✓ q r pf r<d with r ≟ 0
  d ∣? a | no d≢0 | Euclidean✓ q r pf r<d | yes ≡-refl =
    yes $ divides q $ begin
      a ≡⟨ pf ⟩
      0 + d * q ≡⟨⟩
      d * q ≡⟨ *-comm d q ⟩
      q * d ∎
\end{code}
\begin{code}
  d ∣? a | no d≢0 | Euclidean✓ q r pf r<d | no r≢0 = no ¬d∣a
\end{code}
\begin{code}[hide]
    where
\end{code}
In this $\AgdaInductiveConstructor{no}$ case we must prove that the divisor
dividing the dividend leads to a contradiction. All proofs to a contradiction
start with assuming truth, so we assume that the divisor does divide the
dividend $d \, \AgdaFunction{∣} \, a$. We show this contradiction by producing a
proof that $d \leq r$.
\begin{code}
    ¬d∣a : d ∣ a → ⊥
    ¬d∣a d∣a = contradiction d≤r (<⇒≱ r<d)
\end{code}
\begin{code}[hide]
      where
\end{code}
Our division algorithm returns a proof of correctness namely that $a = r + dq$.
We substitute $a$ with $r + dq$ in our assumption of $d \mid a$. Then we remove
the $d * q$ term from the sum as $d$ trivially divides it. Lastly we transform
our divisibility relation into an inequality as the remainder is not zero.
This illustrates the contradiction and the decision procedure is complete.
\begin{code}
      d∣r+d*q : d ∣ r + d * q
      d∣r+d*q = ∣-respʳ pf d∣a

      d∣r : d ∣ r
      d∣r = ∣n+m∣m⇒∣n d∣r+d*q m∣m*n

      d≤r : d ≤ r
      d≤r = ∣⇒≤ r≢0 d∣r
\end{code}
\section{Investigating Definitions}
\label{sec:investigating-definitions}
There are many isomorphic definitions of primality. Unfortunately not all
definitions are equal in the eyes of \Agda{}. For instance, consider the
following definition, ``A prime number is a natural number greater than $1$ that
has no prime divisors smaller then itself.''. This definition translates
directly into \Agda{} as shown below.
\begin{code}[hide]
  module Primality₁ where
    {-# NO_POSITIVITY_CHECK #-}
\end{code}
\begin{code}
    record IsPrime (p : ℕ) : Set where
      inductive
      constructor IsPrime✓
      field
        1<p : 1 < p
        ∀i[i∤p] : ∀ {i} → i < p → IsPrime i → i ∤ p
\end{code}
This definition has one critical flaw, it is refused by \Agda{}. The type is
recursively defined but the recursion is in \textit{negative} position
\cite{harper}. This means the recursion occurs in the domain of a function type.
Allowing recursive calls in negative position makes your logic inconsistent.
Consider the following code and note that $\AgdaDatatype{Bad}$ has a recursive
call in negative position.
\begin{code}[hide]
  {-# NO_POSITIVITY_CHECK #-}
\end{code}
\begin{code}
  data Bad : Set where
    Bad✓ : (Bad → ⊥) → Bad

  Bad-false : Bad → ⊥
  Bad-false (Bad✓ b) = b (Bad✓ b)

  Bad-true : Bad
  Bad-true = Bad✓ Bad-false

  false : ⊥
  false = Bad-false Bad-true
\end{code}
Note how similar $\AgdaFunction{false}$ is to the non-terminating
\textit{omega combinator} shown in \ref{eqn:omega}. The only difference is the required
wrapping and unwrapping of $\AgdaDatatype{Bad}$ values. Unlike the type
$\AgdaDatatype{Bad}$, the type $\AgdaDatatype{IsPrime}$ given above is actually a safe
use of negative recursion. Convincing \Agda{} of this fact is quite hard and
outside the scope of this thesis.
\begin{align}
  \label{eqn:omega}
  \Omega = (\lambda x. \, x \, x) (\lambda x. \, x \, x)
\end{align}
We can remove the recursion by considering when a number can divide a prime. If
some number divides a prime then that number is either $1$ or the prime itself.
This leads to a non-recursive definition of primality. A natural number is prime
when it is greater than $1$ and its only divisors are $1$ and itself.
\begin{code}[hide]
  module Primality₂ where
    infixr 3 _or_
    data _or_ (A : Set) (B : Set) : Set where
\end{code}
\begin{code}
    record IsPrime (p : ℕ) : Set where
      constructor IsPrime✓
      field
        1<p : 1 < p
        ∀i∣p[i≡p] : ∀ {i} → i ∣ p → i ≡ 1 or i ≡ p
\end{code}
The definition above is usable, but in the following sections most of the proofs
will know $1 < i$ for relevant instantiations of $i$. Thus the definition
generates many superfluous cases where the user must show $i = 1$ is impossible.
So we bake the fact that $i$ must be greater than $1$ into the definition. This
yields our final definition.
\begin{code}
  record IsPrime (p : ℕ) : Set where
    constructor IsPrime✓
    field
      1<p : 1 < p
      ∀i∣p[i≡p] : ∀ {i} → 1 < i → i ∣ p → i ≡ p
\end{code}
Using this definition we can prove specific natural numbers are prime by hand.
\begin{code}
  3-isPrime : IsPrime 3
  3-isPrime = IsPrime✓ (from-yes (1 <? 3)) 3-primality
\end{code}
\begin{code}[hide]
    where
\end{code}
In order to prove $3$ is a prime we need to show that any divisor $i$ of $3$ is
equal to $3$. To accomplish this we perform a case analysis on $i$. \Agda{} can
infer that $i = 0, 1$ are impossible cases so we do not have to fill them out.
The case $i = 2$ is impossible as $2 \nmid 3$ and we can illustrate this fact by
running the decision procedure for divisibility on $2$ and $3$. Finally any
input of the form $4 + i$ will not divide $3$ as it will be too large.
\begin{code}
    3-primality : ∀ {i} → 1 < i → i ∣ 3 → i ≡ 3
    3-primality {suc (suc zero)} _ 2∣3 = contradiction 2∣3 (from-no (2 ∣? 3))
    3-primality {suc (suc (suc zero))} _ i∣3 = ≡-refl
    3-primality {suc (suc (suc (suc i)))} _ i∣3 = contradiction 3<3 n≮n
\end{code}
\begin{code}[hide]
      where
\end{code}
\begin{code}
      3+i<3 : 3 + i < 3
      3+i<3 = ∣⇒≤ (λ ()) i∣3

      3<3 : 3 < 3
      3<3 = ≤-<-trans m≤m+n 3+i<3
\end{code}
This rote work does outline a decision procedure for primality of a number $n$. Test every
number between $2$ and $n - 1$ for divisibility with $n$. If any of the numbers
divide $n$ return false, but what does falsehood represent in this instance?
In this case it implies that the number is a
composite. So we define another predicate $\AgdaDatatype{IsComposite}$. A number
is composite if it has a smaller prime factor. Note every composite number is
guaranteed by the fundamental theorem of arithmetic to have a a prime factor. 
\begin{code}
  record IsComposite (c : ℕ) : Set where
    constructor IsComposite✓
    field
      p : ℕ
      p<c : p < c
      p-isPrime : IsPrime p
      p∣c : p ∣ c
\end{code}
Notice how proving compositionality reduces to proving primality.
\begin{code}
  6-isComposite : IsComposite 6
  6-isComposite = IsComposite✓ 3 (from-yes (3 <? 6)) 3-isPrime (from-yes (3 ∣? 6))
\end{code}
\section{A Decision Procedure}
\label{sec:a-decision-procedure}
As always we require a couple lemmas before we can get into the heart of the
decision procedure. First there is no prime less than $2$, this fact is almost
baked into the definition. Second we prove that a number $n$ can not be both prime and
composite. As the number is composite it has some prime factor, but as the
number is prime that divisor must be equal to itself. The factor is defined to
be less than itself and we reach a contradiction.
\begin{code}
  ¬prime<2 : ∀ p → p < 2 → ¬ (IsPrime p)
  ¬prime<2 p p<2 (IsPrime✓ 2≤p _) = contradiction 2≤p (<⇒≱ p<2)
\end{code}
\begin{code}
  exclusive : ∀ {n} → IsPrime n → IsComposite n → ⊥
  exclusive {n}
    (IsPrime✓ 1<n ∀i∣n⇒i≡n)
    (IsComposite✓ p p<n (IsPrime✓ 1<p _) p∣n)
    = contradiction p<n p≮n
\end{code}
\begin{code}[hide]
    where
\end{code}
\begin{code}
    p≮n : ¬ (p < n)
    p≮n = <-irrefl (∀i∣n⇒i≡n 1<p p∣n)
\end{code}
As we are working in a proof relevant setting, the types
$\AgdaDatatype{IsComposite} \, n$ and
$\AgdaFunction{¬} (\AgdaDatatype{IsPrime}\, n)$ do not carry the same amount of
information. The first type knows exactly one factor of $n$ while the
second type knows no factors. Therefore instead of writing a decision procedure
for prime or not prime, we write a decision procedure for prime or composite. The
later can be easily transformed into the former as will be shown at the end of the
chapter. Therefore we construct a type that can return a proof of primality or
compositionality.
\begin{code}
  data Primality (n : ℕ) : Set where
    Composite✓ : IsComposite n → Primality n
    Prime✓ : IsPrime n → Primality n
\end{code}
Unfortunately our final $\AgdaDatatype{IsPrime}$ type is difficult to construct.
This is because the type has ``forgotten'' that every number less than itself
can not divide itself. Luckily the original inconsistent definition of primality
remembers the divisibility of all numbers less than itself. We can break the
negative recursion by using the current definition of primality in the negative
position. We create a new return type that has this definition inlined into
itself. Thus the algorithm proceeds as follows. We test all prime divisors of
$n$ and store them in $\AgdaDatatype{Compositionality}$ then we convert a
$\AgdaDatatype{Compositionality}$ into a $\AgdaDatatype{Primality}$.
\begin{code}
  data Compositionality (n : ℕ) : Set where
    Composite✓ : IsComposite n → Compositionality n
    Prime✓ : (∀ {p} → p < n → IsPrime p → p ∤ n) → Compositionality n
\end{code}
In practice this algorithm is actually mutually recursive as in order to generate
a $\AgdaDatatype{Compositionality}$ you need to generate smaller
$\AgdaDatatype{IsPrime}$ proofs.
\begin{code}
  compositionality : ∀ n → 1 < n → Acc _<_ n → Compositionality n
  primality : ∀ n → 1 < n → Acc _<_ n → Primality n
\end{code}
\begin{code}
  compositionality n 1<n (acc downward)
    = loop 2 (from-yes (1 <? 2)) 1<n ⊂-well-founded ¬p<2[p∤n]
\end{code}
\begin{code}[hide]
    where
\end{code}
We store these smaller primes in a list encoded with continuation passing style.
The following code is an analog to the nil list case. As there are no primes
less than $2$, this makes a good empty list.
\begin{code}
    ¬p<2[p∤n] : ∀ {p} → p < 2 → IsPrime p → p ∤ n
    ¬p<2[p∤n] {p} p<2 p-isPrime _ = contradiction p-isPrime (¬prime<2 p p<2)
\end{code}
Next we need to construct an analog to append. If we have a list of prime
divisibility up to but not including $x$ and a proof that $x$ does not
divide $n$. Then we can construct a list of prime divisibility including $x$. This
proof proceeds by comparing $p$ and $x$ using a trichotomous comparison.
\begin{code}
    cons
      : ∀ {x}
      → (IsPrime x → x ∤ n)
      → (∀ {p} → p < x → IsPrime p → p ∤ n)
      → (∀ {p} → p < 1 + x → IsPrime p → p ∤ n)
    cons {x} x-isPrime⇒x∤n ∀p<x⇒p∤n {p} p<1+x p-isPrime p∣n with <-cmp p x
    ... | tri< p<x _ _ = contradiction p∣n (∀p<x⇒p∤n p<x p-isPrime)
    ... | tri≈ _ ≡-refl _ = contradiction p∣n (x-isPrime⇒x∤n p-isPrime)
    ... | tri> _ _ x<p = contradiction (suc-injective-≤ p<1+x) (<⇒≱ x<p)
\end{code}
Following this proof, we construct the main divisibility testing loop. We count upwards from
$x = 2$
until we reach $n$ using the accessibility predicate defined at the end of chapter
\ref{chap:termination}. We recursively test the primality of $x$. If $x$ is
composite then there is no reason to test divisibility as the prime factor of
$x$ will have been tested previously in the loop. If $x$ is prime then we test if
$x$ divides $n$ using the divisibility decision procedure defined earlier in the
chapter. If $x$ divides $n$ then we immediately return with a proof that $n$ has
a prime divisor. Otherwise we add $x$ into the list and recurse.
\begin{code}
    loop
      : ∀ x → 1 < x → (x≤n : x ≤ n) → Acc _⊂_ [ x , n ∣ x≤n ]
      → (∀ {p} → p < x → IsPrime p → p ∤ n)
      → Compositionality n
    loop x 1<x x≤n (acc next) ∀p<x⇒p∤n with n≤m⇒n<m∨n≡m x≤n
    ... | inj₂ ≡-refl = Prime✓ ∀p<x⇒p∤n
    ... | inj₁ x<n with primality x 1<x (downward x<n)
    ...   | Composite✓ x-isComposite
          = loop (1 + x) (<-trans 1<x n<1+n) x<n (next [1+x,n]⊂[x,n]) ∀p<1+x⇒p∤n
            where
            [1+x,n]⊂[x,n] : [ 1 + x , n ∣ x<n ] ⊂ [ x , n ∣ x≤n ]
            [1+x,n]⊂[x,n] = upward n<1+n ≤-refl
            ∀p<1+x⇒p∤n = cons (λ x-isPrime _ → exclusive x-isPrime x-isComposite) ∀p<x⇒p∤n
    ...   | Prime✓ x-isPrime with x ∣? n
    ...     | yes x∣n = Composite✓ (IsComposite✓ x x<n x-isPrime x∣n)
    ...     | no ¬x∣n = loop (1 + x) (<-trans 1<x n<1+n) x<n (next [1+x,n]⊂[x,n]) ∀p<1+x[p∤n]
            where
            [1+x,n]⊂[x,n] : [ 1 + x , n ∣ x<n ] ⊂ [ x , n ∣ x≤n ]
            [1+x,n]⊂[x,n] = upward n<1+n ≤-refl
            ∀p<1+x[p∤n] = cons (λ _ x∣n → ¬x∣n x∣n) ∀p<x⇒p∤n
\end{code}
Now we transform compositionalities into primalities. If the number is
composite then we can simply return the same proof. Otherwise we need to
transform our continuation passing style list into our final primality condition.
\begin{code}
  primality n 1<n wf@(acc downward) with compositionality n 1<n wf
  ... | Composite✓ isComposite = Composite✓ isComposite
  ... | Prime✓ ∀p<n⇒p∤n = Prime✓ (IsPrime✓ 1<n ∀i∣n⇒i≡n)
\end{code}
\begin{code}[hide]
    where
    n≢0 : n ≢ 0
    n≢0 n≡0 = <-irrefl (≡-sym n≡0) (<-trans 0<1+n 1<n)
\end{code}
To do this we must handle any possible divisor $i$ of $n$. First we
trichotomously compare $i$ and $n$. If $i$ is larger than $n$ we immediately run
into a familiar contradiction. If $i$ is equal to $n$ then we are done and we
return the proof that they are equal, and if $i$ is less than $n$ we
recursively call our decision procedure on $i$. If $i$ is prime then we have
reached a contradiction as we have a list of all the primes less than $n$ and
they do not divide $n$. If $i$ is composite then it has a prime divisor that
transitively divides $n$ and we have reached a contradiction.
\begin{code}
    ∀i∣n⇒i≡n : ∀ {i} → 1 < i → i ∣ n → i ≡ n
    ∀i∣n⇒i≡n {i} 1<i i∣n with <-cmp i n
    ... | tri> _ _ n<i = contradiction (∣⇒≤ n≢0 i∣n) (<⇒≱ n<i)
    ... | tri≈ _ i≡n _ = i≡n
    ... | tri< i<n _ _ with primality i 1<i (downward i<n)
    ...   | Prime✓ i-isPrime
          = contradiction i∣n (∀p<n⇒p∤n i<n i-isPrime)
    ...   | Composite✓ (IsComposite✓ p p<i p-isPrime p∣i)
          = contradiction (∣-trans p∣i i∣n) (∀p<n⇒p∤n (<-trans p<i i<n) p-isPrime)
\end{code}
After ensuring that our number is greater than $1$, we can invoke
$\AgdaFunction{primality}$ and finish our primality decision procedure.
\begin{code}
  prime? : ∀ n → Dec (IsPrime n)
  prime? n with 1 <? n
  ... | no ¬1<n = no λ { (IsPrime✓ 1<n _) → ¬1<n 1<n }
  ... | yes 1<n with primality n 1<n <-well-founded
  ... | Prime✓ isPrime = yes isPrime
  ... | Composite✓ isComposite = no λ { isPrime → exclusive isPrime isComposite }
\end{code}

\end{document}
