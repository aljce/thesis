\documentclass[./Thesis.tex]{subfiles}
\begin{document}

\chapter{Agda Briefly}
\label{chap:agda-briefly}

\epigraph{
  The world of mathematics is becoming very large, the complexity of
  mathematics is becoming very high, and there is a danger of an
  accumulation of mistakes.
}{Vladimir Voevodsky \cite{voevodsky-quote}}

The goal of this chapter is two-fold. The main goal is to introduce the reader
to Agda and it's style of dependently typed proofs. The secondary goal is to
introduce the reader to the syntax and proof styles used in this thesis. Luckily
these goals are complementary so we will tackle them simultaneously.

\section{Functions and Datatypes}
\label{sec:functions-and-datatypes}

The code snippet below defines the standard boolean datatype ($\mathbb{Z}/[2]$).
The keyword \AgdaKeyword{data} allows users to define their own types.
You can think about the type $\AgdaPrimitiveType{Set}$ as the type of all types\footnotemark.
The following lines introduce the common names for boolean values while
expressing that they are of type boolean.
This example also highlights that names in Agda need not be ASCII identifiers.
They can be arbitrary unicode strings namely $\AgdaDatatype{𝔹}$.
\footnotetext{
  This allows for a type theoretic analog of Russell's
  Paradox. In actuality
  $\AgdaPrimitiveType{Set}_{n} : \AgdaPrimitiveType{Set}_{n + 1}$
  but as everything in this thesis lives in
  $\AgdaPrimitiveType{Set}_0 = \AgdaPrimitiveType{Set}$
  we can ignore this complexity.
}
\begin{code}
  data 𝔹 : Set where
    false : 𝔹
    true  : 𝔹

  not : 𝔹 → 𝔹
  not false = true
  not true  = false

  not-example : 𝔹
  not-example = not false
\end{code}
The function $\AgdaFunction{not}$ is defined by case analysis and follows
the standard mathematical definition closely. Functions in \Agda{} must be
total. In other words, they must cover every possible value in their domain.
The function defined above clearly satisfies this condition. Note that function
application, as seen in $\AgdaFunction{not-example}$ is simply an empty space
separating the function and its argument. This convention first appeared in
Alonzo Church's \textit{lambda calculus} and as \Agda{} is an
extension of the lambda calculus the convention carries over. Investigating the
lambda calculus is a thesis worthy exercise and in the past many reed theses
have attempted this challenge. In our case we refer the interested reader to the
following sources \cite{harper} \cite{hott-book}. This being said we consider the
lambda calculus to be a formalism of the abstract mapping notation common when
describing functions in mathematics. Note the following similarities between the
mathematical notation and the lambda notation described in \ref{eqn:mapping}.
\begin{align}
  \label{eqn:mapping}
  x \mapsto x^2 \equiv λ x \, → \, x^2
\end{align}
\Agda{} supports arbitrary lambda expressions and we can use this fact to define
the identity function for booleans in lambda notation.
\begin{code}[hide]
  module Identity₁ where
\end{code}
\begin{code}
    id : 𝔹 → 𝔹
    id = λ x → x
\end{code}
The following identity function is definitionally equal to the lambda function
described above as every function in \Agda{} is desugared to the lambda calculus.
\begin{code}[hide]
  module Identity₂ where
\end{code}
\begin{code}
    id : 𝔹 → 𝔹
    id x = x
\end{code}
The astute reader may find the definitions above suspicious. Why is the identity
function defined with a domain and codomain of the booleans? Does \Agda{}
require a programmer to create a new identity function for every type?
Thankfully \Agda{} adopts a solution pervasive to mathematics, universal
quantification.
\begin{code}[hide]
  module Identity₃ where
\end{code}
\begin{code}
    id : ∀ (A : Set) → A → A
    id A x = x

    id-𝔹 : 𝔹 → 𝔹
    id-𝔹 = id 𝔹
\end{code}
The general identity function depicted above introduces two new concepts.
It introduces universal quantification, or type theoretically, dependent types.
A dependent type comes in two parts. The type of the second half $A \, → \, A$ depends
on the input of the first $∀ (A : \AgdaPrimitiveType{Set}) →$. This is powerful
innovation and we will come to see that it drastically increases language expressivity.
It also introduces function currying. This concept allows multi-argument
functions to be encoded as functions that return functions.
\begin{align}
  \label{eqn:currying}
  (x, y) \mapsto x^2 + y^2 \equiv λ x \, → λ y \, → \, x^2 + y^2
\end{align}
Every function in \Agda{} is automatically curried so the identity function
described above is definitionally equal to the following function.
\begin{code}[hide]
  module Identity₄ where
\end{code}
\begin{code}
    id : ∀ (A : Set) → A → A
    id = λ A → λ x → x
\end{code}
Notice how the dependent type behaves similar to normal function types.
This is because all function types in \Agda{} are actually dependently typed.
The syntax $A \, → \, B$ is just syntactic sugar for a dependent type that ignores
the type of the input $(\_ : A) \, → \, B$. Lastly we make specifying the type
of the input optional as \Agda{} can almost always infer that type. Note that we
can still specify the type as seen in $\AgdaFunction{id-example₂}$ and
$\AgdaFunction{id-example₃}$
\begin{code}[hide]
  module Identity₅ where
\end{code}
\begin{code}
    id : ∀ {A : Set} → A → A
    id {A} x = x

    id-example₁ : 𝔹
    id-example₁ = id true

    id-example₂ : 𝔹
    id-example₂ = id {𝔹} true

    id-example₃ : 𝔹
    id-example₃ = id {A = 𝔹} true
\end{code}
\section{Baby's First Proofs}
\label{sec:first-proofs}
In order to prove interesting mathematical statements we first need a notion of
equality. The following code defines a datatype that represents equality. The
first line informs \Agda{} that equality is a binary relation over an arbitrary
type. The datatype only has a single constructor which requires that the left
hand side is the ``same'' as the right hand side. Thus the constructor is named
$\AgdaInductiveConstructor{≡-refl}$ as these constraints make the constructor
similar to the reflexivity axiom of equivalence classes.
\begin{code}[hide]
  module Equality where
\end{code}
\begin{code}
    data _≡_ {A : Set} : A → A → Set where
      ≡-refl : ∀ {x : A} → x ≡ x
\end{code}
In practice the left hand side and right hand side need not be the exact same
code. They do need to simplify to the same value. In the following code
$\AgdaFunction{not} \, \AgdaInductiveConstructor{true}$ simplifies to
$\AgdaInductiveConstructor{false}$ so the reflexivity constructor is allowed.
\begin{code}
    simple : not true ≡ false
    simple = ≡-refl
\end{code}
These concepts guide the following proof where we prove that $\AgdaFunction{not}$ is an
involution. The proof proceeds by cases, when the input is
$\AgdaInductiveConstructor{false}$ the expression
$\AgdaFunction{not} \,
(\AgdaFunction{not} \, \AgdaInductiveConstructor{false})$
simplifies to $\AgdaInductiveConstructor{false}$ via repeated applications of
the definition of $\AgdaFunction{not}$. The process proceeds symmetrically when the input is
$\AgdaInductiveConstructor{true}$. Note that the case split is required as
$\AgdaFunction{not}$ does not simplify for arbitrary booleans. This is unlike
the identity function which would simplify for any input.
\begin{code}
    not-involution : ∀ (b : 𝔹) → not (not b) ≡ b
    not-involution false = ≡-refl
    not-involution true = ≡-refl
\end{code}
\section{Natural Numbers}
\label{sec:natural-numbers}
The set of mathematical statements provable with only finite constructions is
significantly smaller than the set of all provable mathematical statements.
Thankfully \Agda{} can express many infinite constructions\footnotemark{}.
Infinite constructions are defined by an \textit{inductively defined datatype}
\cite{agda}.
In the following code we define the simplest inductively defined datatype, the Peano
natural numbers. This definition is similar to our definition of the booleans
but the $\AgdaInductiveConstructor{suc}$ constructor recursively
contains another natural.
\footnotetext{It can easily construct any ordinal less than
  $\varepsilon_0$ and with effort can express larger ordinals.
  \cite{ordinals}
}
\begin{code}[hide]
  module Naturals where
    open import Relation.Binary.PropositionalEquality
      using (_≡_; _≢_; module ≡-Reasoning)
      renaming (refl to ≡-refl; sym to ≡-sym; cong to ≡-cong; cong₂ to ≡-cong₂)
    open ≡-Reasoning
\end{code}
\begin{code}
    data ℕ : Set where
      zero : ℕ
      suc  : ℕ → ℕ

    three : ℕ
    three = suc (suc (suc zero))
\end{code}
We can consume these naturals by \textit{pattern matching} \cite{agda}. In the
following example the variable $m$ is bound by pattern matching and is equal to
the input $n$ minus one. This is because we are ``pealing'' off one successor
constructor.
\begin{code}
    isZero : ℕ → 𝔹
    isZero zero = true
    isZero (suc m) = false
\end{code}
\begin{code}[hide]
    infixl 7 _+_
\end{code}
The example above is not a compelling use of natural numbers. The code sample
below defines addition on Peano naturals a more compelling use. As our definition was inductively
defined any non trivial operation on naturals must be inductively defined as
well.
\begin{code}
    _+_ : ℕ → ℕ → ℕ
    zero + n = n
    suc n + m = suc (n + m)
\end{code}
The function name above was defined with mixfix syntax. This is a syntax unique
to \Agda{} that allows for the construction of complex mathematical operators. A
valid name may be interspersed with underscores. At every underscore \Agda{}
expects a user to supply one argument. Addition is a binary operator so we
supply one underscore on either side of the addition symbol. \\

This is a proof assistant so lets prove a simple property of the naturals,
namely that addition is associative. In order to do this we need a definition of
associativity. The following definition is a function that takes
an arbitrary binary operation and returns a specification of associativity for
that binary operation. Note that even arguments can be defined with mixfix.
\begin{code}
    Associative : ∀ {A : Set} → (A → A → A) → Set
    Associative {A} _∙_ = ∀ (x y z : A) → (x ∙ (y ∙ z)) ≡ ((x ∙ y) ∙ z)
\end{code}
The actual proof begins by cases. When the
first argument is zero, $\AgdaInductiveConstructor{zero} + (y + z)$ and
$(\AgdaInductiveConstructor{zero} + y) + z$ both simplify to $y + z$
definitionally. Thus we can invoke reflexivity. The second case is more
complicated and is proved using equational reasoning combinators. The first
combinator $\AgdaFunction{\_≡⟨⟩\_}$ can only be used when the two lines are
definitionally equal to each other. This is the reflexivity axiom in disguise.
The next combinator we use is $\AgdaFunction{\_≡⟨\_⟩\_}$ and it allows us to
supply a proof that the first line equals the second.
We invoke the inductive hypothesis and wrap a
$\AgdaInductiveConstructor{suc}$ around the result. This is allowable as our
equality type is congruent over any function.
Lastly we push the $\AgdaInductiveConstructor{suc}$
inside the term. This is the second line in the definition of addition applied
in reverse.
\begin{code}
    +-assoc : Associative _+_
    +-assoc zero y z = ≡-refl
    +-assoc (suc x) y z = begin
      suc x + (y + z)   ≡⟨⟩
      suc (x + (y + z)) ≡⟨ ≡-cong suc (+-assoc x y z) ⟩
      suc ((x + y) + z) ≡⟨⟩
      (suc (x + y) + z) ≡⟨⟩
      (suc x + y) + z   ∎
\end{code}
\section{Record Types}
\label{sec:record-types}
We introduce one last critical construct, the record construction. A record is
very similar to a datatype but there are some differences. Most importantly
records have only one constructor so they can not be used to define any of the
datatype given above. Unlike datatypes, records contain a list of named fields.
We define a record type that bundles together all the proofs required for a
binary operator to be a semigroup. In this case their is only one constraint
namely associativity. \\
\begin{code}
    record IsSemigroup {A : Set} (_∙_ : A → A → A) : Set where
      constructor IsSemigroup✓
      field
        ∙-assoc : Associative _∙_
\end{code}
We prove that $\AgdaFunction{\_+\_}$ is a semigroup but supplying every
field of the $\AgdaDatatype{IsSemigroup}$ record.
\begin{code}
    +-isSemigroup : IsSemigroup _+_
    +-isSemigroup = IsSemigroup✓ +-assoc
\end{code}
Importantly the type of fields can depend on the value of fields previously in
the list. This allows us to define the less than or equal to relation $n \leq m$
as $\exists k. \, n + k = m$. The reason fields act similarly to exists
that when we construct a record we ``forget'' everything specific to the value
except the requirements specified by the fields.
\begin{code}
    record _≤_ (n : ℕ) (m : ℕ) : Set where
      constructor lte
      field
        k : ℕ
        pf : n + k ≡ m
\end{code}

% \begin{code}
%   Reflexive : ∀ {A : Set} → (A → A → Set) → Set
%   Reflexive _R_ = ∀ {x} → x R x

%   ≡-reflexive : ∀ {A : Set} → Reflexive {A = A} _≡_
%   ≡-reflexive = ≡-refl
% \end{code}
% \begin{code}
%   Symmetric : ∀ {A : Set} → (A → A → Set) → Set
%   Symmetric _R_ = ∀ {x y} → x R y → y R x

%   ≡-sym : ∀ {A : Set} → Symmetric {A = A} _≡_
%   ≡-sym {A} {x} {.x} ≡-refl = ≡-refl
% \end{code}
% Agda supports mixfix syntax for defining names (datatypes, functions, and arguments).
% As shown below anywhere an underscore is given an argument should be supplied.
% \begin{code}
%   Transitive : ∀ {A : Set} → (A → A → Set) → Set
%   Transitive _R_ = ∀ {x y z} → x R y → y R z → x R z

%   ≡-trans : ∀ {A : Set} → Transitive {A = A} _≡_
%   ≡-trans {A} {x} {.x} {.x} ≡-refl ≡-refl = ≡-refl
% \end{code}
% \begin{code}
%   ≡-cong : ∀ {A B : Set} (f : A → B) {x y : A} → x ≡ y → f x ≡ f y
%   ≡-cong {A} {B} f {x} {.x} ≡-refl = ≡-refl
% \end{code}
% \begin{code}
%   infix  1 begin_
%   begin_ : ∀ {A : Set} {x y : A} → x ≡ y → x ≡ y
%   begin x≡y = x≡y

%   infix  3 _∎
%   _∎ : ∀ {A : Set} (x : A) → x ≡ x
%   x ∎ = ≡-refl

%   infixr 2 _≡⟨_⟩_
%   _≡⟨_⟩_ : ∀ {A : Set} (x : A) {y z : A} → x ≡ y → y ≡ z → x ≡ z
%   x ≡⟨ x≡y ⟩ y≡z = ≡-trans x≡y y≡z

%   infixr 2 _≡⟨⟩_
%   _≡⟨⟩_ : ∀ {A : Set} {y : A} (x : A) → x ≡ y → x ≡ y
%   x ≡⟨⟩ ≡-refl = ≡-refl
% \end{code}


\end{document}
