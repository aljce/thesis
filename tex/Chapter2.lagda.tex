\documentclass[./Thesis.tex]{subfiles}
\begin{document}

\chapter{Polynomials}
\label{chap:polynomials}

\begin{code}[hide]
open import AKS.Algebra.Bundles using (DecField)
\end{code}
\begin{code}
module Chapter2 {c ℓ} (F : DecField c ℓ) where
  open DecField F
    using (0#; 1#; _+_; _*_; -_; _-_; _⁻¹; _/_; C/0)
    renaming (Carrier to C)
\end{code}

Before we can prove facts about polynomials, we first consider how to represent
polynomials as a datastructure. You could represent polynomials as a list of
coefficient's and exponents. The way mathematicians think about polynomials but
this representation has a glaring flaw. Its not normalized by construction. This
problem appears when considering adding the polynomial
$x^2 + 2x^2 = [ \, (1, 2), \, (2, 2) \, ]$ 
and $x^2 = [ \, (1, 2) \, ]$. In this example to which coefficient do you add
the second polynomial's coefficient? The common solution to this problem is too
normalize before every operation. 

\begin{AgdaAlign}
  \begin{code}
  data Polynomialⁱ : Set c where
    0ⁱ    : Polynomialⁱ
    _+x∙_ : C → Polynomialⁱ → Polynomialⁱ
  \end{code}
  \begin{align*}
    x^3 + 2x + 1 = 1 + x\left( 2 + x\left( 0 + x\left( 1 + x\left( 0 \right) \right) \right) \right)
  \end{align*}
  \begin{code}
  1ⁱ : Polynomialⁱ
  1ⁱ = 1# +x∙ 0ⁱ
  \end{code}
  \begin{code}[hide]
  infixl 6 _+ⁱ_
  \end{code}
  \begin{code}
  _+ⁱ_ : Polynomialⁱ → Polynomialⁱ → Polynomialⁱ
  0ⁱ +ⁱ q = q
  (c₁ +x∙ p) +ⁱ 0ⁱ = c₁ +x∙ p
  (c₁ +x∙ p) +ⁱ (c₂ +x∙ q) = (c₁ + c₂) +x∙ (p +ⁱ q)
  \end{code}
  \begin{code}[hide]
  infix 8 x∙_
  \end{code}
  \begin{code}
  x∙_ : Polynomialⁱ → Polynomialⁱ
  x∙ p = 0# +x∙ p
  \end{code}
  \begin{code}[hide]
  infixl 7 _∙ⁱ_
  \end{code}
  \begin{code}
  _∙ⁱ_ : C → Polynomialⁱ → Polynomialⁱ
  a ∙ⁱ 0ⁱ = 0ⁱ
  a ∙ⁱ (c +x∙ p) = (a * c) +x∙ (a ∙ⁱ p)
  \end{code}
  \begin{code}[hide]
  infixl 7 _*ⁱ_
  \end{code}
  \begin{code}
  _*ⁱ_ : Polynomialⁱ → Polynomialⁱ → Polynomialⁱ
  0ⁱ *ⁱ q = 0ⁱ
  (c₁ +x∙ p) *ⁱ 0ⁱ = 0ⁱ
  (c₁ +x∙ p) *ⁱ (c₂ +x∙ q) = (c₁ * c₂) +x∙ (c₁ ∙ⁱ q +ⁱ c₂ ∙ⁱ p +ⁱ x∙ (p *ⁱ q))
  \end{code}
\end{AgdaAlign}

\end{document}