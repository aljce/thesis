\documentclass[./Thesis.tex]{subfiles}
\begin{document}

\chapter{Exponentiation}
\label{chap:exponentiation}

\epigraph{
  Accurate reckoning. \\
  The entrance into the knowledge of all existing
  things and all obscure secrets.
}{
  Translated from the Rhind Egyptian Mathematical Papyrus.
  \cite{rhind-papyrus}
}

In this chapter we will investigate formalizing exponentiation and the
algorithms to compute it. As \Agda{} is a proof theoretic language these things
are one and the same \cite{hott-book}.

\section{Inductive Definitions}
\label{sec:inductive-definitions}

\begin{code}[hide]
  open import Algebra using (CommutativeMonoid)
  module Exponentiation where
  open import AKS.Nat using (ℕ; zero; suc; _+_; _*_; _<_)
  open import Data.Nat.Properties using (*-identityˡ; *-assoc)
  open import Relation.Binary.PropositionalEquality using (_≡_; _≢_; module ≡-Reasoning) renaming (refl to ≡-refl; sym to ≡-sym; cong to ≡-cong)
\end{code}

Before thinking of formalizing a piece of mathematics its often useful to step
back and think about the desired semantics of your formalism. In this case we
ask the question what are the semantics of exponentiation. Middle and Elementary
students often learn that exponentiation is ``repeated'' multiplication. This
thought is made more precise below:
\begin{align*}
  x^n = \underbrace{x \times x \times \dots \times x}_{n \text{ multiplications}}
\end{align*}
The equation above lacks parentheses, arbitrarily assigning right precedence
the equation becomes:
\begin{align*}
  x^n = x \times (x \times (x \times \dots))
\end{align*}
This leads to an obvious inductive definition to compute $x^n$ just multply $x$
by $x^{n -1}$ with a base case of $x^0 = 1$. This definition cleanly bakes a law
of exponents $x^{1+n} = x^1 \times x^n = x \times x^n$ into the definition. This
idea is formalized below. \\

\begin{code}[hide]
  module Basic (C : Set) where
    open ≡-Reasoning using (begin_; _≡⟨_⟩_; _≡⟨⟩_; _∎)
    infixr 8 _^_
\end{code}
\begin{code}[inline]
    _^_ : ℕ → ℕ → ℕ
    x ^ zero = 1
    x ^ suc n = x * x ^ n
\end{code} \\

We demonstrate the correctness of this definition by proving one of the
fundemental laws of exponentiation,
the exponentiational homomorphism $x^{n + m} = x^n x^m$. \\
\begin{code}
    ^-homo : ∀ x n m → x ^ (n + m) ≡ x ^ n * x ^ m
\end{code} \\
The type above just makes the quantifiers explict. The base case below
is more intresting. We first simplify and unwrap the definitons then
tac a $1$ onto the exponential. This is allowable as $1$ is the
multpliciative unit. Finally we rewrap definitions, notice that
$x \string^{} \, 0$ is defined to be $1$ so the simplifier can work in reverse.
\begin{code}
    ^-homo x zero m = begin
      x ^ (0 + m)   ≡⟨⟩
      x ^ m         ≡⟨ ≡-sym (*-identityˡ (x ^ m)) ⟩
      1 * x ^ m     ≡⟨⟩
      x ^ 0 * x ^ m ∎
\end{code} \\
The inductive step starts similarlly to the base case in so far as we
unfold the defintions. Then we invoke the inductive hypothesis which in \Agda{}
is just a recursive call to our proof. Notice that the induction is well founded
as the natural $n$ is decreasing by one each inductive step.
Lastly we associate the multiplications correctly.
\begin{code}
    ^-homo x (suc n) m = begin
      x ^ (suc n + m)     ≡⟨⟩
      x ^ suc (n + m)     ≡⟨⟩
      x * x ^ (n + m)     ≡⟨ ≡-cong (λ e → x * e) (^-homo x n m) ⟩
      x * (x ^ n * x ^ m) ≡⟨ ≡-sym (*-assoc x (x ^ n) (x ^ m)) ⟩
      (x * x ^ n) * x ^ m ≡⟨⟩
      (x ^ suc n) * x ^ m ∎
\end{code} \\

\end{document}
