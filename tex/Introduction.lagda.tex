\documentclass[./Thesis.tex]{subfiles}
\begin{document}

\chapter{Agda Briefly}
\label{chap:agda-briefly}

The goal of this chapter is two-fold. The main goal is to introduce the reader
to Agda and it's style of dependently typed proofs. The secondary goal is to
introduce the reader to the syntax and proof styles used in this thesis. Luckily
these goals are complementary so we will tackle them simultaneously.

\section{Functions and Datatypes}
\label{sec:functions-and-datatypes}

The code snipit below defines the standard boolean datatype ($\mathbb{Z}/[2]$).
The keyword \AgdaKeyword{data} allows users to define their own types.
You can think about the type \Set as the type of all small types.
The following lines introduce constants that have type $\Boolean$.
This example also highlights that names in Agda need not be ASCII identifiers.
They can be arbitrary unicode strings.
\begin{code}
  data 𝔹 : Set where
    false : 𝔹
    true  : 𝔹

  not : 𝔹 → 𝔹
  not false = true
  not true  = false

  example : 𝔹
  example = not false
\end{code}
The function $\AgdaFunction{not}$ is defined by case analysis and follows
the standard mathematical definition closely. Functions in agda must be
total. They must cover every possible value in their domain.
The function defined above clearly satisfies this condition. \\
Agda supports mixfix syntax for defining names (datatypes and functions).
As shown below anywhere an underscore is given an argument should be supplied.
\CitationNeeded{Agda}
\begin{code}
  if_then_else_ : ∀ {A : Set} → 𝔹 → A → A → A
  if true  then x else y = x
  if false then x else y = y
\end{code}
The $\AgdaFunction{if\_then\_else\_}$ function introduces two novel
concepts in agda, dependent types and implicit arguments. The figure below
introduces a simple dependent type. The type of $\AgdaFunction{id₁}$ comes
in two parts. The type of the second half $A → A$ depends on the type of the first $∀ (A : \Set) →$.
In mathematics universal instantiation is often an axiom but in Agda you instantiate the
forall in the same way as you would any Agda function. The example
$\AgdaFunction{id₁-example₁}$ demonstrates instantiation in Agda.
\begin{code}
  id₁ : ∀ (A : Set) → A → A
  id₁ A x = x

  id₁-example₁ : 𝔹
  id₁-example₁ = id₁ 𝔹 true
\end{code}
The following code demonstrates an implicit dependent type. This functions the
same as above except you do not need to specify the first argument. Agda will
perform unification to infer the type. This can be seen in
$\AgdaFunction{id₂-example₁}$ where $A = \Boolean$ as
$\AgdaInductiveConstructor{true} : \Boolean$. The other examples show that you
can still explicitly instantiate the implicit forall. This is quite useful for
when the type is ambiguous or Agda's unifier fails.
\begin{code}
  id₂ : ∀ {A : Set} → A → A
  id₂ {A} x = x

  id₂-example₁ : 𝔹
  id₂-example₁ = id₂ true

  id₂-example₂ : 𝔹
  id₂-example₂ = id₂ {𝔹} true

  id₂-example₃ : 𝔹
  id₂-example₃ = id₂ {A = 𝔹} true
\end{code}

\begin{code}
  infixr 2 _∷_
  data List (A : Set) : Set where
    []  : List A
    _∷_ : A → List A → List A

  list-example : List 𝔹
  list-example = true ∷ false ∷ true ∷ []

  map : ∀ {A B : Set} → (A → B) → List A → List B
  map f [] = []
  map f (x ∷ xs) = f x ∷ map f xs

  list-example₂ : List 𝔹
  list-example₂ = map not list-example
\end{code}

\begin{code}
  data _≡_ {A : Set} : A → A → Set where
    refl : ∀ {x : A} → x ≡ x

  simple : (if true then false else true) ≡ false
  simple = refl

  not-involution : ∀ (b : 𝔹) → not (not b) ≡ b
  not-involution false = refl
  not-involution true = refl
\end{code}

\begin{code}
  Symmetric : ∀ {A : Set} → (A → A → Set) → Set
  Symmetric _R_ = ∀ {x y} → x R y → y R x

  sym : ∀ {A : Set} → Symmetric {A = A} _≡_
  sym {A} {x} {.x} refl = refl

  Transitive : ∀ {A : Set} → (A → A → Set) → Set
  Transitive _R_ = ∀ {x y z} → x R y → y R z → x R z

  trans : ∀ {A : Set} → Transitive {A = A} _≡_
  trans {A} {x} {.x} {.x} refl refl = refl

  cong : ∀ {A B : Set} (f : A → B) {x y : A} → x ≡ y → f x ≡ f y
  cong {A} {B} f {x} {.x} refl = refl
\end{code}

\section{Natural Numbers}
\label{sec:natural-numbers}

The natural numbers, in their simplicity, offer a clean path to explore the
foundations of mathematics. As noted 

\begin{code}
  infix  1 begin_
  begin_ : ∀ {A : Set} {x y : A} → x ≡ y → x ≡ y
  begin x≡y = x≡y

  infix  3 _∎
  _∎ : ∀ {A : Set} (x : A) → x ≡ x
  x ∎ = refl

  infixr 2 _≡⟨_⟩_
  _≡⟨_⟩_ : ∀ {A : Set} (x : A) {y z : A} → x ≡ y → y ≡ z → x ≡ z
  x ≡⟨ x≡y ⟩ y≡z = trans x≡y y≡z
\end{code}

\begin{code}
  data ℕ : Set where
    zero : ℕ
    suc  : ℕ → ℕ

  isZero : ℕ → 𝔹
  isZero zero = true
  isZero (suc n) = false

  infixl 7 _+_
  _+_ : ℕ → ℕ → ℕ
  zero + n = n
  suc n + m = suc (n + m)

  Associative : ∀ {A : Set} → (A → A → A) → Set
  Associative {A} _∙_ = ∀ (x y z : A) → (x ∙ (y ∙ z)) ≡ ((x ∙ y) ∙ z)

  record IsSemigroup {A : Set} (_∙_ : A → A → A) : Set where
    field
      ∙-assoc : Associative _∙_

  +-assoc : Associative _+_
  +-assoc zero y z = refl
  +-assoc (suc x) y z = begin
    suc x + (y + z)   ≡⟨ refl ⟩
    suc (x + (y + z)) ≡⟨ cong suc (+-assoc x y z) ⟩
    suc ((x + y) + z) ≡⟨ refl ⟩
    (suc (x + y) + z) ≡⟨ refl ⟩
    (suc x + y) + z   ∎

  +-isSemigroup : IsSemigroup _+_
  +-isSemigroup = record { ∙-assoc = +-assoc }

  record _×_ (A : Set) (B : Set) : Set where
    constructor _,_
    field
      fst : A
      snd : B

  LeftIdentity : ∀ {A : Set} → A → (A → A → A) → Set _
  LeftIdentity {A} ε _∙_ = ∀ (x : A) → (ε ∙ x) ≡ x

  RightIdentity : ∀ {A : Set} → A → (A → A → A) → Set _
  RightIdentity {A} ε _∙_ = ∀ (x : A) → (x ∙ ε) ≡ x

  Identity : ∀ {A : Set} → A → (A → A → A) → Set _
  Identity ε ∙ = (LeftIdentity ε ∙) × (RightIdentity ε ∙)

  record IsMonoid {A : Set} (ε : A) (_∙_ : A → A → A) : Set where
    field
      ∙-isSemigroup : IsSemigroup _∙_
      ∙-identity : Identity ε _∙_

  +-identityˡ : LeftIdentity zero _+_
  +-identityˡ x = refl

  +-identityʳ : RightIdentity zero _+_
  +-identityʳ zero = refl
  +-identityʳ (suc x) = begin
    (suc x + zero) ≡⟨ refl ⟩
    suc (x + zero) ≡⟨ cong suc (+-identityʳ x) ⟩
    suc x          ∎
    
  +-identity : Identity zero _+_
  +-identity = ( +-identityˡ , +-identityʳ )

  +-isMonoid : IsMonoid zero _+_
  +-isMonoid = record { ∙-isSemigroup = +-isSemigroup ; ∙-identity = +-identity }
\end{code}

\begin{code}
\end{code}

\end{document}
