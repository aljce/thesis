\documentclass[./Thesis.tex]{subfiles}
\begin{document}

\chapter{Introduction}
\label{chap:Introduction}
\section{AKS}

\begin{algorithm}[H]
  \caption{The AKS Primality Decision Algorithm}
  \label{fig:prime-slow}
  \begin{algorithmic}
    \Function{IsPrime}{$n$}
    \State $r \leftarrow 2$
    \While{true}
      \If{$gcd(r, n) = 1$}
        \State \textbf{return} NO
      \EndIf
      \State $q \leftarrow$ largest prime factor of $r - 1$
      \If{$q \geq 2 \sqrt{r} \, \Lg{n} + 2$ and
          $n^{(r - 1) / q} \not\equiv 1 \text{ mod } r$}
        \State \textbf{break}
      \EndIf
      \State $r \leftarrow$ next prime after $r$
    \EndWhile
    \For{$a \leftarrow 1$ \textbf{to} $\Floor[\big]{2 \sqrt{r} \, \Lg{n}} + 1$}
      \If{$(x - a)^n \not\equiv (x^n - a) \text{ mod } (x^r - 1) \text{ in } \mathbb{Z}_n[x]$}
        \State \textbf{return} NO
      \EndIf
    \EndFor
    \If{$n$ is a perfect power}
      \State \textbf{return} NO
    \Else
      \State \textbf{return} YES
    \EndIf
    \EndFunction
  \end{algorithmic}
\end{algorithm}
% defined well typed
% define setoid
% talk about the limitations of agda
\section{Type Theory}
In the rest of this Chapter we provide an informal motivation of Type Theory for a reader
familiar with the standard foundations of mathematics, Set Theory. Our proof
assistant \Agda{} is a proof assistant based on Type Theory. For readers that
desire a more formal specification of Type Theory we recommend the following
sources: a topological perspective \cite{hott-book}, a categorical perspective
\cite{nlab}, and a programming language theory perspective \cite{harper}.
\subsection{History}
\label{sec:history}
Type Theory was created by Bertrand Russell as an alternate foundation for
mathematics in order to, quash the plethora of
paradoxes plaguing Set Theory at the turn of the century. The paradoxes
where eventually resolved and type theory faded into obscurity. It resurfaced in
the 1940's when Alonzo Church applied it to his \textit{lambda calculus}. This
application began a reassurance in which multiple authors steadily improved the
expressivity of Type Theory. This culminated with Per Martin-L\"{o}f's
dependent, intuitionistic type theory often called
\textit{Martin-L\"{o}f Type Theory} \cite{martin-lof}.
This theory is considered by many as the foundations of the field. The
programming language \Agda{} that we will come to know in this thesis is a
variant of his Type Theory. This being said the field is vast and there are many
Type Theories created with wildly different axioms and properties in mind.
\subsection{The Typing Relation}
\label{sec:typing-relation}
The primitive concept underlying Type Theory is the \textit{typing relation},
namely a \textit{term} $a$ has \textit{type} $A$ written $a : A$. In the
language of Set Theory this would represent ``a is an element of the set A''
denoted $a \in A$. Despite these similarities types differ from sets in many
ways. For instance the perfectly valid set
$\{ \, \text{true}, \, 0, \, \emptyset \, \}$
is inexpressible in Type Theory as types are ``homogeneous''. More importantly
there is not an equivalent concept of a non-membership ($\notin$) in most Type
Theories. Instead we say $2$ does not have type boolean or we say $2$ is not
well typed as a boolean. Thus while its sensible to ask the set theoretic question
``Is $a$ a member of $A$ for some pre-existing objects $a$ and $A$?''. Type
theoretically we are not permitted to consider objects without their type and
every object has a type from creation. For instance we can define the booleans
($𝔹$) as two constants $\text{true} : 𝔹$ and $\text{false} : 𝔹$.
\subsection{Propositions as Types}
In reality Set Theory is actually a dual theory. There is a base theory called
first order logic in which a collection of axioms are developed that describe
the structure of sets. So Set Theory is really the study of these two
interleaved theories. Type Theory is its own logic which collapses the two
levels into one. The two basic elements of Set Theory, logical propositions and
sets thus also collapse together. They collapse together into types and
propositions become manipulatable objects. For instance
a function $f$ is injective if the following proposition is true.
\begin{align}
  \text{for any } x, y \text{ if } f(x) = f(y) \text{ then } x = y
\end{align}
We can represent this proposition type theoretically with the following
expression. 
\begin{equation}
  \label{eqn:injective}
  \text{Injective}(f : A \to B) :\equiv \prod_{x : A} \prod_{y : A} \, f(x) \equiv_B f(y) \to x \equiv_A y
\end{equation}
Now in order to prove some function $f$ is injective we simply construct a proof
that has the type specified above, for instance $p : \text{Injective}(f)$.
Thus proving a proposition is simply finding an element of some type.
Also note that proofs are values and so they can be
manipulated like any other value. This concept is called
\textit{proof relevance} and it is essential to Type Theory and this thesis.
\subsection{Intuitionistic Logic}
In this section we investigate the deep consequences of proof relevance.
Consider the Collatz conjecture, concerning the sequence that follows, start with some
integer $n$ and add $n$ to the sequence. If $n$ is even then divide by two
otherwise multiply $n$ by $3$ and add $1$. The result of this computation is the
next element of the sequence. Repeat the process above. The conjecture is
that for any starting value $n$ the sequence eventually reaches $1$. This
conjecture is a well known unsolved problem in mathematics and as such the truth
value of the proposition is unknown. Now consider the following proposition
often called the law of the excluded middle.
\begin{align}
  \label{eqn:set-LEM}
  \text{for any proposition } p \text{ either } p \text{ or } \neg p
\end{align}
We can transform this proposition into Type Theory just as we have before. Note
that propositions are types so we can easily quantify over every proposition by
quantifying over every type.
\begin{align}
  \label{eqn:set-LEM}
  \text{LEM} :\equiv \prod_{p  : \text{Type}} p \vee \neg p
\end{align}
In Set Theory the law of the excluded middle is assumed as an axiom of first
order logic. Presumably Type Theorists could assume it as well and there would
be some value $\text{lem} : \text{LEM}$ representing the axiom. This axiom behaves weirdly
with proof relevance in mind. Consider using this axiom to construct a proof
that the Collatz conjecture is true or false. We simply instantiate $p$ to our
type theoretic encoding of the Collatz conjecture. Thus if we inspect this proof
term we can learn if the Collatz conjecture is true or false. As solving math
problems is not this easy we must not assume the law of the excluded middle.
This variant of logic is called \textit{intuitionistic logic}. \\

This being said the law of the excluded middle appears often
and is a useful axiom. Thankfully the axiom is often provable when you choose a
specific $p$. Imagine instantiating $p$ with the proposition
$n \equiv_{\mathbb{N}} m$ for some $n$ and $m$.
The resulting type is described below.
\begin{align}
  \label{eqn:nat-dec}
  \text{Equals?} :\equiv (n \equiv_{\mathbb{N}} m) \vee \neg (n \equiv_{\mathbb{N}} m)
\end{align}
This is the type of our first decision procedure. A value of this type would
prove that the two numbers are equal or prove that the two numbers are unequal.
Finding a value of this type is as simple as describing a decision procedure.
Simply count $n$ and $m$ down by $1$. If both numbers reach $0$ at the same time
return that they are equal otherwise return that they are unequal. \\

Keep in mind that assuming the law of the excluded middle also invalidates the
entire purpose of this thesis. We can define a proposition that represents
primality (and in fact we do in Chapter \ref{chap:primality}) apply the law of
the excluded middle to this type and now we have a term that ``computes''
primality.
\end{document}
