\documentclass[12pt,twoside]{reedthesis}
\usepackage{amsmath}
\usepackage{amssymb}
\usepackage{amsthm}
\usepackage{mathbbol}
\usepackage{mathtools}
\usepackage{stmaryrd}
\DeclarePairedDelimiter\Parens{\lparen}{\rparen}
\DeclarePairedDelimiter\Floor{\lfloor}{\rfloor}
\usepackage{gauss}

\usepackage{subfiles}
\usepackage{epigraph}
\setlength{\epigraphwidth}{0.9\textwidth}
\usepackage{xcolor}

\usepackage{tikz}
\usepackage{tikz-qtree}

\usepackage{algorithm}
\usepackage[noend]{algpseudocode}

\usepackage{url}
\usepackage{hyperref}
\usepackage[T1]{fontenc}
\usepackage[utf8]{inputenc}
\usepackage[links]{agda}
\usepackage{newunicodechar}
\newunicodechar{λ}{\ensuremath{\mathnormal\lambda}}
\newunicodechar{→}{\ensuremath{\mathnormal\to}}
\newunicodechar{∀}{\ensuremath{\mathnormal\forall}}
\newunicodechar{𝔹}{\ensuremath{\mathbb{B}}}
\newunicodechar{ℕ}{\ensuremath{\mathbb{N}}}
\newunicodechar{₁}{\ensuremath{{}_1}}
\newunicodechar{¹}{\ensuremath{{}^1}}
\newunicodechar{₂}{\ensuremath{{}_2}}
\newunicodechar{²}{\ensuremath{{}^2}}
\newunicodechar{₃}{\ensuremath{{}_3}}
\newunicodechar{³}{\ensuremath{{}^3}}
\newunicodechar{₄}{\ensuremath{{}_4}}
\newunicodechar{⁴}{\ensuremath{{}^4}}
\newunicodechar{ᵇ}{\ensuremath{{}^b}}
\newunicodechar{⁺}{\ensuremath{{}^+}}
\newunicodechar{≡}{\small {\ensuremath{\equiv}}}
\newunicodechar{≢}{\small {\ensuremath{\not\equiv}}}
\newunicodechar{≟}{\small {\ensuremath{\stackrel{?}{=}}}}
\newunicodechar{∙}{\ensuremath{\cdot}}
\newunicodechar{×}{\ensuremath{\times}}
\newunicodechar{ε}{\ensuremath{\varepsilon}}
\newunicodechar{∎}{\ensuremath{\rule{1ex}{1.2ex}}}
\newunicodechar{⟨}{\ensuremath{\langle}}
\newunicodechar{⟩}{\ensuremath{\rangle}}
\newunicodechar{ˡ}{\ensuremath{{}^l}}
\newunicodechar{ʳ}{\ensuremath{{}^r}}
\newunicodechar{∷}{\ensuremath{::}}
\newunicodechar{ⁱ}{\ensuremath{{}^i}}
\newunicodechar{ʰ}{\ensuremath{{}^h}}
\newunicodechar{ℓ}{\ensuremath{\ell}}
\newunicodechar{⁻}{\ensuremath{{}^{-}}}
\newunicodechar{≈}{\ensuremath{\approx}}
\newunicodechar{𝕓}{\ensuremath{\mathbb{b}}} % TODO: This looks terrible
\newunicodechar{⇓}{\ensuremath{\Downarrow}}
\newunicodechar{⇑}{\ensuremath{\Uparrow}}
\newunicodechar{⇒}{\ensuremath{\Rightarrow}}
\newunicodechar{⇐}{\ensuremath{\Leftarrow}}
\newunicodechar{⟦}{\ensuremath{\llbracket}}
\newunicodechar{⟧}{\ensuremath{\rrbracket}}
\newunicodechar{✓}{\ensuremath{\checkmark}}
\newunicodechar{∣}{\ensuremath{\mid}}
\newunicodechar{∤}{\ensuremath{\nmid}}
\newunicodechar{¬}{\ensuremath{\neg}}
\newunicodechar{⊥}{\ensuremath{\bot}}
\newcommand{\Dotdiv}{%
  \mathbin{%
    \vphantom{+}%
    \text{%
      \mathsurround=0pt%
      \ooalign{%
        \noalign{\kern-.35ex}%
        \hidewidth$\smash{\cdot}$\hidewidth\cr%
        \noalign{\kern.35ex}%
        $-$\cr%
      }%
    }%
  }%
}
\newunicodechar{∸}{\ensuremath{\Dotdiv}}
\newunicodechar{≤}{\ensuremath{\leq}}
\newunicodechar{≱}{\ensuremath{\not\geq}}
\newunicodechar{≮}{\ensuremath{\not<}}
\newunicodechar{≺}{\ensuremath{\prec}}
\newunicodechar{⊂}{\ensuremath{\subset}}
\newunicodechar{∨}{\ensuremath{\wedge}}

\newcommand{\Agda}[0]{Agda}

\newcommand{\BigO}[1]{\ensuremath{O\Parens{#1}}}

\newcommand{\Lg}[1]{\ensuremath{\text{lg}\Parens{#1}}}

\newcommand{\Line}[1]{{\color{#1} \strut\vrule}}

\setlength{\parindent}{0pt}

\title{A Replication of the AKS Primality Decision Algorithm}
\author{Alice J. McKean}
\date{May 2020}
\division{Mathematical and Natural Sciences}
\advisor{James D. Fix}
\department{Computer Science}

\begin{document}

\maketitle
\frontmatter
\pagestyle{empty}

\chapter*{Acknowledgments}
Jim, I'm so grateful for your sage advice and unyielding patience. You routinely
knew exactly what needed to be said. \\

Friends, you know who you are, thank you for listening to my rambling ravings
and supporting me through this journey.

\chapter*{Abstract}
In this thesis we explore formalizing the AKS primality decision algorithm in
the \Agda{} proof assistant. We begin with a discussion of the AKS algorithm, an
algorithm that determines if a given input is prime or composite in polynomial time.
This algorithm will guide the development of basic number theory in the \Agda{} programming
language. This language is a dependently typed formal logic capable of ensuring
the correctness of mathematical statements. In this mathematical assembly we
prove the correctness of the exponentiation by squaring algorithm. Then we
explore the nature of recursive algorithms by convincing \Agda{} that their
recursion is well founded and terminating. Lastly we provide a fully
formalized brute force primality decision procedure.

\chapter*{Dedication}
\textit{To my family, eternal love}

\tableofcontents

\mainmatter
\pagestyle{fancyplain}

\subfile{Introduction}
\subfile{TypeTheory}
\subfile{AgdaBriefly}
\subfile{Exponentiation}
\subfile{Termination}
\subfile{Primality}
\subfile{Conclusion}

\appendix

\backmatter

\nocite{gnu-parallel}
\bibliographystyle{apalike}
\bibliography{citations}

\end{document}
